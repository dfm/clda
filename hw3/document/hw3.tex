\documentclass[11pt]{article}
\usepackage{fullpage}
\usepackage{fancyhdr}

\usepackage{amsmath}
\usepackage{amssymb}
\usepackage{url}

\usepackage{listings}
\usepackage{color}
\lstset{language=Python,
        basicstyle=\footnotesize\ttfamily,
        showspaces=false,
        showstringspaces=false,
        tabsize=2,
        breaklines=false,
        breakatwhitespace=true,
        identifierstyle=\ttfamily,
        keywordstyle=\color[rgb]{0,0,1},
        commentstyle=\color[rgb]{0.133,0.545,0.133},
        stringstyle=\color[rgb]{0.627,0.126,0.941},
    }

\usepackage[pdftex]{graphicx}

% header
\fancyhead{}
\fancyfoot{}
\fancyfoot[C]{\thepage}
\fancyhead[R]{Daniel Foreman-Mackey}
\fancyhead[L]{Statistical Natural Language Processing --- Homework 3}
\pagestyle{fancy}
\setlength{\headsep}{10pt}
\setlength{\headheight}{20pt}

% shortcuts
\newcommand{\Eq}[1]{Equation (\ref{eq:#1})}
\newcommand{\eq}[1]{Equation (\ref{eq:#1})}
\newcommand{\eqlabel}[1]{\label{eq:#1}}
\newcommand{\Fig}[1]{Figure~\ref{fig:#1}}
\newcommand{\fig}[1]{Figure~\ref{fig:#1}}
\newcommand{\figlabel}[1]{\label{fig:#1}}

\newcommand{\pr}[1]{\ensuremath{p\left (#1 \right )}}
\newcommand{\lk}[1]{\ensuremath{\mathcal{L} \left ( #1 \right )}}
\newcommand{\bvec}[1]{\ensuremath{\boldsymbol{#1}}}
\newcommand{\dd}{\ensuremath{\, \mathrm{d}}}
\newcommand{\normal}[2]{\ensuremath{\mathcal{N} \left ( #1; #2 \right ) }}
\newcommand{\T}{^\mathrm{T}}

\newcommand{\data}{\mathcal{D}}
\newcommand{\code}[1]{{\sffamily #1}}


\begin{document}

As with the last assignment, I decided to implement my homework in Python
instead of using the provided Java code.
I was surprised to find that the language modeling implemented in pure-Python
was very computationally efficient.
This is because the data structures in the standard library are pretty
impressive!
I also started implementing a version that looked a bit more like the provided
code (using a lot of objects) and that was painfully slow.
I wasn't able to make the Viterbi algorithm bearable in pure-Python so I
dropped down to C for that part (but I still evaluate the trigram/emission
model using Python).
All of the code that I used for this assignment is available on GitHub:
\url{https://github.com/dfm/nlp}.

\section{Baseline Model}

\end{document}
